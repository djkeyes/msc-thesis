\graphicspath{{chapter1/}}

\chapter{Chapter1}
\label{cha:chapter1}

This is the first chapter of the template thesis of the PIXHAWK project.\\

\begin{figure}[htbp]
	\centering
		\includegraphics[width=0.60\textwidth]{pixhawk-logo.png}
	\caption{The PIXHAWK project logo \cite{pix_wiki}}
	\label{fig:pixhawk-logo}
\end{figure}

\section{The PIXHAWK Project}
\label{sec:pixhawkproject}

PIXHAWK is a student project team of the Computer Vision and Geometry (CVG) Lab at the ETH Zurich. The goal of the PIXHAWK project is to build a helicopter that will be able to explore indoor and outdoor environments autonomously and identify objects or classes of objects. Typical usage scenarios are visual inspection of air conditioning systems, and disaster recovery. In disaster recovery, the helicopter will enter semi-collapsed buildings and search for victims needing assistance. Due to the small size and the non-surface-touching drive mechanism, the helicopter can enter locations not accessible by ground robots, dogs or humans.

\cleardoublepage
